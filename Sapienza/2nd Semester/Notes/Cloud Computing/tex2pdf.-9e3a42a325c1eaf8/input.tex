% Options for packages loaded elsewhere
\PassOptionsToPackage{unicode}{hyperref}
\PassOptionsToPackage{hyphens}{url}
%
\documentclass[
]{article}
\usepackage{amsmath,amssymb}
\usepackage{iftex}
\ifPDFTeX
  \usepackage[T1]{fontenc}
  \usepackage[utf8]{inputenc}
  \usepackage{textcomp} % provide euro and other symbols
\else % if luatex or xetex
  \usepackage{unicode-math} % this also loads fontspec
  \defaultfontfeatures{Scale=MatchLowercase}
  \defaultfontfeatures[\rmfamily]{Ligatures=TeX,Scale=1}
\fi
\usepackage{lmodern}
\ifPDFTeX\else
  % xetex/luatex font selection
\fi
% Use upquote if available, for straight quotes in verbatim environments
\IfFileExists{upquote.sty}{\usepackage{upquote}}{}
\IfFileExists{microtype.sty}{% use microtype if available
  \usepackage[]{microtype}
  \UseMicrotypeSet[protrusion]{basicmath} % disable protrusion for tt fonts
}{}
\makeatletter
\@ifundefined{KOMAClassName}{% if non-KOMA class
  \IfFileExists{parskip.sty}{%
    \usepackage{parskip}
  }{% else
    \setlength{\parindent}{0pt}
    \setlength{\parskip}{6pt plus 2pt minus 1pt}}
}{% if KOMA class
  \KOMAoptions{parskip=half}}
\makeatother
\usepackage{xcolor}
\usepackage{color}
\usepackage{fancyvrb}
\newcommand{\VerbBar}{|}
\newcommand{\VERB}{\Verb[commandchars=\\\{\}]}
\DefineVerbatimEnvironment{Highlighting}{Verbatim}{commandchars=\\\{\}}
% Add ',fontsize=\small' for more characters per line
\newenvironment{Shaded}{}{}
\newcommand{\AlertTok}[1]{\textcolor[rgb]{1.00,0.00,0.00}{\textbf{#1}}}
\newcommand{\AnnotationTok}[1]{\textcolor[rgb]{0.38,0.63,0.69}{\textbf{\textit{#1}}}}
\newcommand{\AttributeTok}[1]{\textcolor[rgb]{0.49,0.56,0.16}{#1}}
\newcommand{\BaseNTok}[1]{\textcolor[rgb]{0.25,0.63,0.44}{#1}}
\newcommand{\BuiltInTok}[1]{\textcolor[rgb]{0.00,0.50,0.00}{#1}}
\newcommand{\CharTok}[1]{\textcolor[rgb]{0.25,0.44,0.63}{#1}}
\newcommand{\CommentTok}[1]{\textcolor[rgb]{0.38,0.63,0.69}{\textit{#1}}}
\newcommand{\CommentVarTok}[1]{\textcolor[rgb]{0.38,0.63,0.69}{\textbf{\textit{#1}}}}
\newcommand{\ConstantTok}[1]{\textcolor[rgb]{0.53,0.00,0.00}{#1}}
\newcommand{\ControlFlowTok}[1]{\textcolor[rgb]{0.00,0.44,0.13}{\textbf{#1}}}
\newcommand{\DataTypeTok}[1]{\textcolor[rgb]{0.56,0.13,0.00}{#1}}
\newcommand{\DecValTok}[1]{\textcolor[rgb]{0.25,0.63,0.44}{#1}}
\newcommand{\DocumentationTok}[1]{\textcolor[rgb]{0.73,0.13,0.13}{\textit{#1}}}
\newcommand{\ErrorTok}[1]{\textcolor[rgb]{1.00,0.00,0.00}{\textbf{#1}}}
\newcommand{\ExtensionTok}[1]{#1}
\newcommand{\FloatTok}[1]{\textcolor[rgb]{0.25,0.63,0.44}{#1}}
\newcommand{\FunctionTok}[1]{\textcolor[rgb]{0.02,0.16,0.49}{#1}}
\newcommand{\ImportTok}[1]{\textcolor[rgb]{0.00,0.50,0.00}{\textbf{#1}}}
\newcommand{\InformationTok}[1]{\textcolor[rgb]{0.38,0.63,0.69}{\textbf{\textit{#1}}}}
\newcommand{\KeywordTok}[1]{\textcolor[rgb]{0.00,0.44,0.13}{\textbf{#1}}}
\newcommand{\NormalTok}[1]{#1}
\newcommand{\OperatorTok}[1]{\textcolor[rgb]{0.40,0.40,0.40}{#1}}
\newcommand{\OtherTok}[1]{\textcolor[rgb]{0.00,0.44,0.13}{#1}}
\newcommand{\PreprocessorTok}[1]{\textcolor[rgb]{0.74,0.48,0.00}{#1}}
\newcommand{\RegionMarkerTok}[1]{#1}
\newcommand{\SpecialCharTok}[1]{\textcolor[rgb]{0.25,0.44,0.63}{#1}}
\newcommand{\SpecialStringTok}[1]{\textcolor[rgb]{0.73,0.40,0.53}{#1}}
\newcommand{\StringTok}[1]{\textcolor[rgb]{0.25,0.44,0.63}{#1}}
\newcommand{\VariableTok}[1]{\textcolor[rgb]{0.10,0.09,0.49}{#1}}
\newcommand{\VerbatimStringTok}[1]{\textcolor[rgb]{0.25,0.44,0.63}{#1}}
\newcommand{\WarningTok}[1]{\textcolor[rgb]{0.38,0.63,0.69}{\textbf{\textit{#1}}}}
\setlength{\emergencystretch}{3em} % prevent overfull lines
\providecommand{\tightlist}{%
  \setlength{\itemsep}{0pt}\setlength{\parskip}{0pt}}
\setcounter{secnumdepth}{-\maxdimen} % remove section numbering
\ifLuaTeX
  \usepackage{selnolig}  % disable illegal ligatures
\fi
\IfFileExists{bookmark.sty}{\usepackage{bookmark}}{\usepackage{hyperref}}
\IfFileExists{xurl.sty}{\usepackage{xurl}}{} % add URL line breaks if available
\urlstyle{same}
\hypersetup{
  hidelinks,
  pdfcreator={LaTeX via pandoc}}

\author{}
\date{}

\begin{document}

\hypertarget{part-1-getting-started}{%
\section{2.1.1 Part 1: Getting started}\label{part-1-getting-started}}

\hypertarget{q1-why-you-run-the-container-dockergetting-started-in-detached-mode}{%
\subsection{Q1: Why you run the container docker/getting-started in
detached
mode?}\label{q1-why-you-run-the-container-dockergetting-started-in-detached-mode}}

Detached mode means that the container will be running in the
\textbf{background}, without user interaction. This is useful because
the terminal where the process was started can closed and the user can
run other commands.

\hypertarget{q2-what-is-the-difference-between-a-container-and-a-container-image}{%
\subsection{Q2: What is the difference between a container and a
container
image?}\label{q2-what-is-the-difference-between-a-container-and-a-container-image}}

The container image is the file that contains all of the instructions
for how to \textbf{build} the container. It is the blueprint that tells
docker how to create the container. Depending on you needs, there are
many images available depending on your needs, e.g.~a python image that
comes with python already installed. The container is a \textbf{runnable
instance} of the image which is managed by Docker.

\hypertarget{part-2-sample-application}{%
\section{2.1.2 Part 2: Sample
application}\label{part-2-sample-application}}

\hypertarget{q1-what-is-the-meaning-of-the-dockers-directives-used-in-the-docker-file-please-comment-on-each-of-the-directives.}{%
\subsection{Q1: What is the meaning of the docker’s directives used in
the docker file? Please comment on each of the
directives.}\label{q1-what-is-the-meaning-of-the-dockers-directives-used-in-the-docker-file-please-comment-on-each-of-the-directives.}}

\begin{Shaded}
\begin{Highlighting}[]
\CommentTok{\# }\CommentVarTok{syntax=docker/dockerfile:1}

\CommentTok{\# specifies the base image that we will build our custom image around}
\KeywordTok{FROM}\NormalTok{ node:18{-}alpine }

\CommentTok{\# sets the working directory for other instructions like COPY. E.g. when referring to the current directory, it will be in /app.}
\KeywordTok{WORKDIR}\NormalTok{ /app}

\CommentTok{\# copies everything from the current host directory, into the /app directory.}
\KeywordTok{COPY}\NormalTok{ . .}

\CommentTok{\# runs the "yarn install {-}{-}production" command, which installs dependencies}
\KeywordTok{RUN} \ExtensionTok{yarn}\NormalTok{ install }\AttributeTok{{-}{-}production}

\CommentTok{\# the command that is executed when the container is started. Here, it will run the node app}
\KeywordTok{CMD}\NormalTok{ [}\StringTok{"node"}\NormalTok{, }\StringTok{"src/index.js"}\NormalTok{]}

\CommentTok{\# opens up port 3000 on the container}
\KeywordTok{EXPOSE}\NormalTok{ 3000}
\end{Highlighting}
\end{Shaded}

\hypertarget{q2-why-it-is-important-to-tag-a-container-image}{%
\subsection{Q2: Why it is important to tag a container
image?}\label{q2-why-it-is-important-to-tag-a-container-image}}

Tagging images is important because it allows you to \textbf{identify}
your image after building it. If you do not specify a tag, you will have
to use the \textbf{image ID} whenever you interact with container,
e.g.~stopping, starting etc. As the number of images in your repository
grows, identifying which image does what becomes increasingly difficult.
Additionally, with tags you can specify \textbf{versions} of your image,
where each tag is a version number, e.g.~\texttt{0.0.1}, that points to
a specific version.

\hypertarget{q3-why-we-should-bind-a-host-port-with-the-container-port}{%
\subsection{Q3: Why we should bind a host port with the container
port?}\label{q3-why-we-should-bind-a-host-port-with-the-container-port}}

Binding a host port with a container port forwards any requests that
come to the host port to the specified container port. This is import
for applications that are running inside the container and arre
listening for incoming messages on that port. For example, a
containerised webserver that is listening on port \texttt{5000}, will
only receive messages if it is bound to host port.

\hypertarget{part-3-update-the-application}{%
\section{2.1.3 Part 3: Update the
application}\label{part-3-update-the-application}}

\hypertarget{q1-it-is-possible-to-bind-two-containers-on-the-same-host-port}{%
\subsection{Q1: It is possible to bind two containers on the same host
port?}\label{q1-it-is-possible-to-bind-two-containers-on-the-same-host-port}}

No, it is not possible to bind two containers to the same host port.
Each host port can only be assigned to a \textbf{single} process.

\hypertarget{q2-why-and-when-after-stopping-a-container-you-could-need-to-remove-it}{%
\subsection{Q2: Why and when, after stopping a container, you could need
to remove
it?}\label{q2-why-and-when-after-stopping-a-container-you-could-need-to-remove-it}}

You remove containers to free up space on your system.

\hypertarget{q3-it-is-possible-to-remove-a-running-container-without-stopping-it-before-the-removal}{%
\subsection{Q3: It is possible to remove a running container, without
stopping it before the
removal?}\label{q3-it-is-possible-to-remove-a-running-container-without-stopping-it-before-the-removal}}

No, to be able to remove a container it first needs to be stopped.
Though you can stop and remove it with a \textbf{single} command.

\hypertarget{part-4-share-the-application}{%
\section{2.1.4 Part 4: Share the
application}\label{part-4-share-the-application}}

\hypertarget{q1-given-a-container-image-available-on-a-docker-image-repository-can-you-start-an-instance-of-the-image-on-any-docker-host-is-there-any-limitation}{%
\subsection{Q1: Given a container image available on a docker image
repository, can you start an instance of the image on any docker host?
Is there any
limitation?}\label{q1-given-a-container-image-available-on-a-docker-image-repository-can-you-start-an-instance-of-the-image-on-any-docker-host-is-there-any-limitation}}

Yes, it should be possible to start the instance of the image on
\textbf{any} docker host. The whole idea of docker is to separate the
application from the host that is running it by shipping the host itself
so that it can run on any system with docker installed.

\hypertarget{part-5-persist-the-db}{%
\section{2.1.5 Part 5: Persist the DB}\label{part-5-persist-the-db}}

\hypertarget{q1-if-you-run-two-instances-of-the-same-container-image-lets-call-them-container-a-and-container-b-and-you-create-a-file-in-container-a-is-that-new-file-visible-in-container-b}{%
\subsection{Q1: If you run two instances of the same container image,
let’s call them container A and container B, and you create a file in
container A, is that new file visible in container
B?}\label{q1-if-you-run-two-instances-of-the-same-container-image-lets-call-them-container-a-and-container-b-and-you-create-a-file-in-container-a-is-that-new-file-visible-in-container-b}}

If all you do is to create the file in container A, it will \textbf{not}
be visible in container B since their file systems are separated.

\hypertarget{q2-why-in-the-docker-command-docker-run--d-ubuntu-bash--cshuf--i-1-10000--n-1--o-data.txt-tail--f-devnull-we-need-to-keep-the-container-running-with-tail--f-devnull}{%
\subsection{Q2: Why in the docker command “docker run -d ubuntu bash
-c”shuf -i 1-10000 -n 1 -o /data.txt \&\& tail -f /dev/null” we need to
keep the container running with “tail -f
/dev/null”?}\label{q2-why-in-the-docker-command-docker-run--d-ubuntu-bash--cshuf--i-1-10000--n-1--o-data.txt-tail--f-devnull-we-need-to-keep-the-container-running-with-tail--f-devnull}}

The command \texttt{tail\ -f\ /dev/null} watches the \texttt{data.txt}
file so that the container keeps running after \texttt{data.txt} file
has been created for us to inspect it.

\hypertarget{q3-what-you-can-do-with-the-docker-exec-command}{%
\subsection{Q3: What you can do with the “docker exec”
command?}\label{q3-what-you-can-do-with-the-docker-exec-command}}

The \texttt{docker\ exec} command let’s you execute a command inside the
container. E.g.

\begin{verbatim}
docker exec -it <container ID> bash
\end{verbatim}

runs the \texttt{bash} command inside the container.

\hypertarget{q4-let-assume-you-decide-to-use-a-volume.-why-you-need-to-mount-the-volume-in-the-container-file-system-does-that-mean-you-modify-the-container-filesystem}{%
\subsection{Q4: Let assume you decide to use a volume. Why you need to
mount the volume in the container file system? Does that mean you modify
the container
filesystem?}\label{q4-let-assume-you-decide-to-use-a-volume.-why-you-need-to-mount-the-volume-in-the-container-file-system-does-that-mean-you-modify-the-container-filesystem}}

The volume lets you connect a specific path on the container file system
to the host file system and persist data from the container on the host.
We need to mount the volume into the container file system so that the
volume data becomes available inside the container. This means that the
container filesystem is modified, since any data that is in the volume,
now lives inside the container as well.

\hypertarget{q5-in-this-part-of-the-tutorial-you-have-created-a-volume.-where-is-the-volume-located-in-the-file-system-of-your-docker-host}{%
\subsection{Q5: In this part of the tutorial, you have created a volume.
Where is the volume located in the file system of your docker
host?}\label{q5-in-this-part-of-the-tutorial-you-have-created-a-volume.-where-is-the-volume-located-in-the-file-system-of-your-docker-host}}

The volume is located at \texttt{/var/lib/docker/volumes/} on the docker
host.

\hypertarget{part-6-use-bind-mounts}{%
\section{2.1.6 Part 6: Use bind mounts}\label{part-6-use-bind-mounts}}

\hypertarget{q1-what-is-a-dev-mode-container}{%
\subsection{Q1: What is a dev-mode
container?}\label{q1-what-is-a-dev-mode-container}}

A dev-mode container is a container which has the development workspace
from the host \textbf{mounted} into it. This means that we can write
code on the host machine and immediately see the results in the
application that is running inside the container.

\hypertarget{q2-can-we-run-a-container-install-software-dependencies-and-then-use-the-updated-container-without-building-first-the-image}{%
\subsection{Q2: Can we run a container, install software dependencies,
and then use the updated container without building first the
image?}\label{q2-can-we-run-a-container-install-software-dependencies-and-then-use-the-updated-container-without-building-first-the-image}}

\hypertarget{part-7-multi-container-app-part-8-use-docker-compose}{%
\section{2.1.7 Part 7: Multi container app \& Part 8: Use Docker
Compose}\label{part-7-multi-container-app-part-8-use-docker-compose}}

\hypertarget{q1-what-is-a-docker-service}{%
\subsection{Q1: What is a Docker
service?}\label{q1-what-is-a-docker-service}}

\hypertarget{q2-can-we-spin-up-a-single-instance-of-a-docker-container-using-a-docker-compose-file}{%
\subsection{Q2: Can we spin up a single instance of a docker container
using a docker-compose
file?}\label{q2-can-we-spin-up-a-single-instance-of-a-docker-container-using-a-docker-compose-file}}

Yes, we can specify as many services as we want inside the
\texttt{docker-compose.yaml} file.

\hypertarget{q3-can-we-run-a-mysql-container-and-store-the-database-structure-and-data-in-a-volume}{%
\subsection{Q3: Can we run a MySQL container and store the database
structure and data in a
volume?}\label{q3-can-we-run-a-mysql-container-and-store-the-database-structure-and-data-in-a-volume}}

Yes, we can. The MySQL database is stored in a \texttt{.db} file, so all
we need to do is to create a volume and mount this file when starting
the MySQL container.

\hypertarget{q4-is-the-order-of-the-services-defined-in-a-docker-compose-file-important-or-is-it-irrelevant-which-service-is-defined-first}{%
\subsection{Q4: Is the order of the services defined in a docker-compose
file important, or is it irrelevant which service is defined
first?}\label{q4-is-the-order-of-the-services-defined-in-a-docker-compose-file-important-or-is-it-irrelevant-which-service-is-defined-first}}

The order of service definition has no impact on the order in which they
are started and will be ready, so if we do not care about starting
services in a specific order, we can define them in any order we like.

\hypertarget{q5-is-it-mandatory-to-define-the-network-in-a-docker-compose-file}{%
\subsection{Q5: Is it mandatory to define the network in a
docker-compose
file?}\label{q5-is-it-mandatory-to-define-the-network-in-a-docker-compose-file}}

No, it is not mandatory to define the network in the
\texttt{docker-compose.yaml}. Docker compose will automatically create a
network with the service names as aliases, if we do not specify one.

\hypertarget{q6-if-you-would-like-to-run-a-multi-container-app-is-it-necessary-to-use-docker-compose-i.e.-to-define-a-service-or-you-can-achieve-the-same-objective-using-docker-commands-from-the-shell-does-a-service-offers-more-than-just-running-a-multiple-container-app-with-a-single-command}{%
\subsection{Q6: If you would like to run a multi-container app, is it
necessary to use docker compose (i.e.~to define a service) or you can
achieve the same objective using docker commands from the shell? Does a
service offers more than just running a multiple container app with a
single
command?}\label{q6-if-you-would-like-to-run-a-multi-container-app-is-it-necessary-to-use-docker-compose-i.e.-to-define-a-service-or-you-can-achieve-the-same-objective-using-docker-commands-from-the-shell-does-a-service-offers-more-than-just-running-a-multiple-container-app-with-a-single-command}}

It is not necessary to use docker compose for a multi-container app. One
can achieve the same thing by using docker commands in the command line.
However, using docker compose is a better approach when dealing with
multiple containers because the entire configuration is written down as
code. This means that the configurartion can be versioned, reviewed, and
contributed to through a version control system. Additionally, it is
easy to share and to repeatedly execute without resorting to manual
intervention.

\end{document}
